\documentclass[12pt]{article}

% Pacotes %
\usepackage[brazilian]{babel}
\usepackage[utf8]{inputenc}
\usepackage[T1]{fontenc}
\usepackage{enumitem}
\usepackage{hyperref}
\usepackage{graphicx}

% Definições de titulo e autores %
\title{Iniciação a Computação Científica - Trabalho 2}
\author{
	Giancarlo Klemm Camilo \\
	Renan Domingos Merlin Greca
}
\date{Junho de 2015}

% Inicio do documento %
\begin{document}

% ------------------------------------------------------ %
% Página inicial %
\maketitle
\newpage	

% ------------------------------------------------------ %
% Indice %
\tableofcontents
\newpage

% ------------------------------------------------------ %
\section{Introdução}

\newpage

% ------------------------------------------------------ %
\section{Análise de Arquitetura}

\newpage

% ------------------------------------------------------ %
\section{Limite Superior da Discretização}

\newpage

% ------------------------------------------------------ %
\section{Tempo de Execução}

	\subsection{Programa Original}
	\subsection{Programa Otimizado}
\newpage

% ------------------------------------------------------ %
\section{Análise de Funções}

	\subsection{Programa Original}
		\subsubsection{Método de Gauss-Seidel}
		\subsubsection{Cálculo do Resíduo}
	
	\subsection{Programa Otimizado}
		\subsubsection{Método de Gauss-Seidel}
		\subsubsection{Cálculo do Resíduo}

	\subsection{Análise dos Dados}

\newpage

% ------------------------------------------------------ %
\section{Otimização do Ponto de Interesse}

\newpage

% ------------------------------------------------------ %
% Otimizações %
\section{Otimizações}

\subsection{Estrutura de dados}

\subsection{Código}

\newpage

% ------------------------------------------------------ %
% Resultados %
\section{Resultados}

\subsection{Tempo}

\subsection{Memória}

\newpage

\end{document}